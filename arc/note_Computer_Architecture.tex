\documentclass{article}
\usepackage{notes}
\newcommand{\page}[1]{\linkout{book_Computer_Architecture}{#1}}

\title{Computer Architecture}
\begin{document}
\maketitle
\tableofcontents
\newpage

\section{Fundamentals of Quantitative Design and Analysis}
\subsection{Introduction}

\subsection{Classes of Computers}

\subsubsection*{Classes of Parallelism and Parallel Architectures}

\begin{enumerate}
\item Basic classification:
\begin{itemize}
    \item Data-Level Parralelism.
    \item Task-Level Parralelism.
\end{itemize}

\item Classification by exploitation:
\begin{itemize}
    \item Instruction-Level.
    \item Vector Architectures and Graphic Processor Units.
    \item Thread-Level(tightly coupled).
    \item Request-Level(largely decoupled).
\end{itemize}

\item Classification by \textbf{\color{orange} instruction-stream}
\begin{itemize}
    \item SISD.
    \item SIMD. Applying the same operations to multiple items of data in parallel. Mainly for DLP.
    \item MISD. None.
    \item MIMD. Mainly for task-levle parralelism.
\end{itemize}

\end{enumerate}

\subsection{Defining Computer Architecture\page{40}}

{\centering ISA(Instruction Set Architecture) \par}

\begin{enumerate}
    \item Class of ISA.
        \begin{itemize}
            \item \textit{register-memory}: 80x86.
            \item \textit{load-store}: ARM, MIPS.
        \end{itemize}
    \item Memory addressing. \\
    Byte addressing and alignment.\page{531}

    \item Adressing modes. \\
    Register, Immediate, and Displacement(variations).

    \item Types and sizes of operands.

    \item Operations.

    \item Control flow instructions. \\
    Conditional branches, unconditional jumps, procedure calls, returns.
    \item Encoding an ISA. \\
    Fixed length v.s. variable length.

\end{enumerate}

{\centering Designing the Organization and Hardware \par}

\subsection{Trends in Technology}

{\centering Performance Trends: Bandwidth over Latency \page{48} \par}


\end{document}